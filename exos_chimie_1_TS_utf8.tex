\documentclass[a4paper,oneside,10pt]{article}
\usepackage[frenchb]{babel}
\usepackage[utf8]{inputenc}
\usepackage{fancyhdr}
%\usepackage{truncate} % package pour troncature de texte dans l'ent�te
\usepackage{layouts}
\usepackage{array}
\usepackage{calc}
\usepackage{graphicx}
\usepackage{amsmath}
\usepackage{amsthm}
\usepackage{boxedminipage}
\usepackage{chemfig}
\usepackage{siunitx}
%\pagestyle{fancy}
\fancyhf{} % on efface tout
%\lhead{Physique-Chimie. Devoir Seconde  \leftmark} %en tete centr�e � gauche
%\cfoot{\thepage} %num�ro de la page en bas au centre
%% On red�finit le Style des section sousection poura avoir I/1.a)
%\renewcommand\thechapter       {\arabic{chapter}}
\renewcommand\thesection       {\Roman{section}.}
\renewcommand\thesubsection    {\thesection \arabic{subsection}.}
\renewcommand\thesubsubsection {\thesubsection \alph{subsubsection})}
\renewcommand\theparagraph     {\thesubsubsection.\arabic{paragraph}}
\renewcommand\thesubparagraph  {\theparagraph.\arabic{subparagraph}}
\setcounter{secnumdepth}{4}


\textwidth=18.5cm
\textheight=25cm
\marginparsep=5mm %espace marge droite et coprs du texte%
\marginparwidth=55pt
\footskip=40pt
\oddsidemargin= -35pt %espace bord gauche%
\topmargin=-35pt
\headsep=-25pt



%Mise en forme des remarques 
\theoremstyle{remark}
\newtheorem*{rem}{\underline{Remarque}}
\newtheorem*{thm}{\textbf{D�finition}}

\newcommand{\rmq}[1]{\begin{rem} #1 \end{rem} }

% deftt deuxi�eme version de deft avec cr�ation d'un tableau 1case pb reste � d�finir la longueur
% du tableau pour le retour � la ligne
\newcommand{\donnees}[1]{
\newline  \noindent
 \begin{tabular}{|p{15cm}|c|}
\hline 
\textbf{\emph{Donn�es : }}#1\tabularnewline
\hline
\end{tabular}}
% C'est la cr�ation de la commande deft :
% un cadre minipage de 10 cm, environnement th�or�me
% [1] c'est pour l'argument de la fonction \deft
\newcommand{\deft}[1]{\newline \linebreak
\begin{boxedminipage}{150mm} %box sur la largeur de textwidth
\begin{thm}
#1 %contenu de la d�finition
\end{thm}
\end{boxedminipage}
\linebreak %saut de ligne apr�s la d�finition
}
%fin du code de deft

%Qqs nouvelles commandes%
\newcommand{\fiche}[3]{
	\begin{center}
		\textsc{#1}\\
		#2\hfill\textit{#3}
	\end{center}
	\hrule\vspace{\baselineskip}
}
\newcounter{numeroexo}
\newcommand{\exo}{\par\noindent\stepcounter{numeroexo}
	\hspace{-.25cm}\fbox{\textbf{Exercice \arabic{numeroexo}}}\quad}
%fin des nouvelles commandes 

\begin{document}
\fiche{Chimie}{Terminale S en \the\year}{Quelques incontournables}
Les masses molaires sont à chercher dans le livre, les substances utilisées dans ces exercices sont très courantes, vous devez connaître leurs noms ainsi que leurs formules. Reportez vous au cours ou au TP en cas d'oubli.\\
\exo Couples redox
\begin{enumerate}
\item Ecrire les demi équations électroniques pour les couples suivants: $H^+/H_2$, $I_2/I^-$, $MnO_4^-/Mn^{2+}$, $S_2O_3^{2-}/S_4O_6^{2-}$
\item On fait réagit  du diiode $I_2$ sur les ions thiosulfate. Ecrire l'équation chimique de la réaction.
\item  On fait réagit des ions fer $Fe^{2+}$ sur des ions permanganate. Ecrire l'équation chimique de la réaction. $Fe^{2+}$ appartient au couple $Fe^{3+}/Fe^{2+}$.
\end{enumerate}
\exo Calcul de quantité de matière
\begin{enumerate}
\item Soit une quantité d'eau de n=0,068 mol. Calculer la masse d'eau correspondante et le nombre de molécules d'eau dans cet échantillon.
\item 
\end{enumerate}
\exo Faire une solution par dissolution\\
On dispose de chlorure de sodium et on souhaite préparer  50 mL d'une solution d'eau salée dont la concentration est 0,10 mol/L. On appelle $S_1$ cette solution.
\begin{enumerate}
\item De quelle quantité de matière en chlorure de sodium (NaCl) a-t-on besoin pour réaliser la solution $S_1$ ? En déduire la masse de chlorure de sodium à peser.
\item Donner le protocole expérimental permettant de préparer la solution $S_1$.
\end{enumerate}

On souhaite préparer 50 mL d'une solution de sulfate de sodium ($Na_2SO_4$) à 0,20 mol/L. Soit $S_2$ cette solution.
\begin{enumerate}
\item Donner le protocole expérimental qui permet de réaliser $S_2$.
\item Que vaut $[SO_4^{2-}]$ ? Même question pour $[Na^+]$.
\item On mélange 25 mL de $S_1$ et 25 mL de $S_2$, on obtient $S_3$. Faire l'inventaire des ions contenus dans $S_3$.
\item Calculer la concentration de chaque ion contenu dans $S_3$
\end{enumerate}

 \exo Dilution\\
 On dispose d'une solution de permanganate de potassium dont la concentration est $\SI{3,0e-3}{\mol\per\liter}$. Cette solution est appelée solution mère car c'est la solution la plus concentrée dans ce problème. On souhaite préparer 50 mL d'une solution 5 fois moins concentrée en permanganate de potassium.
 \begin{enumerate}
 \item Quelle est la valeur de la concentration de la solution fille obtenue après dilution ?
 \item Imaginons que cette solution fille soit prête... Quelle quantité de matière $n$ en permanganate est contenue dans cette solution fille ?
 \item Cette quantité de matière a été prélevée à l'aide d'une pipette jaugée dans la solution mère. Calculer le volume de solution mère contenant la quantité de matière nécessaire.
 \item En reprenant les résultats précédents rédiger un protocole expérimental expliquant comment préparer la solution fille dont la concentration est 5 fois plus faible que la solution mère.
 \end{enumerate}
 
 \exo
Dilution \emph{bis}
\\
On dispose d'une solution de sulfate de cuivre à $\SI{0.20}{mol/L}$. Donner un protocole expérimental permettant de préparer 50 mL de solution de sulfate de cuivre à $\SI{2.0e-2}{mol/L}$.
\newpage
\exo Transformation chimique et quantité de matière.
La combustion du carbone dans le dioxygène produit uniquement du dioxyde de carbone. L'équation chimique décrivant cette combustion est $$C+O_2\longrightarrow CO_2$$

\begin{enumerate}
\item Identifier les produits et les réactifs.
\item On considère la réaction entre 24 g de carbone et 64 g de dioxygène. Quelle quantité de produit va apparaitre ? Reste-t-il du carbone ou du dioxygène à la fin de la réaction ?
\item On considère la réaction entre 12 g de carbone et 64 g de dioxygène. Lorsque la réaction est terminée, quelles sont les quantités de matières de chaque substance ($C$, $O_2$ et $CO_2$) ? En faisant ce calcul on dit qu'on a fait un bilan de quantité de matière.
\item A l'aide du bilan de matière précédent, indiquez le réactif introduit en excès et le réactif limitant.
\end{enumerate}
On considère maintenant la combustion du fer dans le dioxygène en la modélisant par l'équation chimique suivante:$$3Fe+2O_2 \longrightarrow Fe_3O_4$$
\begin{enumerate}
\item Sans faire de tableau d'avancement, prévoir quelle masse de fer on peut faire bruler à l'aide de 64 g de dioxygène.
\item On fait réagir 55,8 g de fer avec 32 g de dioxygène. Faire le bilan de matière lorsque la réaction est terminée (à l'état final). On pourra faire un tableau d'avancement. On prendra $M_{Fe}=\SI{55,8}{\gram\per\mol}$.
\end{enumerate}
\end{document}













