\documentclass[a4paper,oneside,10pt]{article}
\usepackage[frenchb]{babel}
\usepackage[utf8]{inputenc}
\usepackage{fancyhdr}
%\usepackage{truncate} % package pour troncature de texte dans l'ent�te
\usepackage{layouts}
\usepackage{array}
\usepackage{calc}
\usepackage{graphicx}
\usepackage{amsmath}
\usepackage{amsthm}
\usepackage{boxedminipage}
\usepackage{chemfig}
\usepackage{siunitx}
%\pagestyle{fancy}
\fancyhf{} % on efface tout
%\lhead{Physique-Chimie. Devoir Seconde  \leftmark} %en tete centr�e � gauche
%\cfoot{\thepage} %num�ro de la page en bas au centre
%% On red�finit le Style des section sousection poura avoir I/1.a)
%\renewcommand\thechapter       {\arabic{chapter}}
\renewcommand\thesection       {\Roman{section}.}
\renewcommand\thesubsection    {\thesection \arabic{subsection}.}
\renewcommand\thesubsubsection {\thesubsection \alph{subsubsection})}
\renewcommand\theparagraph     {\thesubsubsection.\arabic{paragraph}}
\renewcommand\thesubparagraph  {\theparagraph.\arabic{subparagraph}}
\setcounter{secnumdepth}{4}


\textwidth=18.5cm
\textheight=25cm
\marginparsep=5mm %espace marge droite et coprs du texte%
\marginparwidth=55pt
\footskip=40pt
\oddsidemargin= -35pt %espace bord gauche%
\topmargin=-35pt
\headsep=-25pt



%Mise en forme des remarques 
\theoremstyle{remark}
\newtheorem*{rem}{\underline{Remarque}}
\newtheorem*{thm}{\textbf{D�finition}}

\newcommand{\rmq}[1]{\begin{rem} #1 \end{rem} }

% deftt deuxi�eme version de deft avec cr�ation d'un tableau 1case pb reste � d�finir la longueur
% du tableau pour le retour � la ligne
\newcommand{\donnees}[1]{
\newline  \noindent
 \begin{tabular}{|p{15cm}|c|}
\hline 
\textbf{\emph{Donn�es : }}#1\tabularnewline
\hline
\end{tabular}}
% C'est la cr�ation de la commande deft :
% un cadre minipage de 10 cm, environnement th�or�me
% [1] c'est pour l'argument de la fonction \deft
\newcommand{\deft}[1]{\newline \linebreak
\begin{boxedminipage}{150mm} %box sur la largeur de textwidth
\begin{thm}
#1 %contenu de la d�finition
\end{thm}
\end{boxedminipage}
\linebreak %saut de ligne apr�s la d�finition
}
%fin du code de deft

%Qqs nouvelles commandes%
\newcommand{\fiche}[3]{
	\begin{center}
		\textsc{#1}\\
		#2\hfill\textit{#3}
	\end{center}
	\hrule\vspace{\baselineskip}
}
\newcounter{numeroexo}
\newcommand{\exo}{\par\noindent\stepcounter{numeroexo}
	\hspace{-.25cm}\fbox{\textbf{Exercice \arabic{numeroexo}}}\quad}
%fin des nouvelles commandes